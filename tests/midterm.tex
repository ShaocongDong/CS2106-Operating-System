\documentclass[12pt,a4paper]{article}

% package importing
%\usepackage[margin=2cm]{geometry}
\usepackage{geometry}
 \geometry{
 a4paper,
 %total={170mm,257mm},
 left=10mm,
 top=20mm,
 right=10mm,
 bottom=20mm,
 }
 		%$$$$$$ fonts settings $$$$$$%
%\usepackage[sc]{mathpazo}  %for palatino font
%\usepackage{eulervm}  %for Euler maths font

		%$$$$$$ package imports $$$$$$%
\usepackage{amsmath}  %for mathematics
\usepackage{titlesec}  %for title spacing only
\usepackage{lipsum}  %random huge text generation
\usepackage{titlesec} %for changing font of titles 
\usepackage{amssymb}  %for real number set symbol
\usepackage{amsthm}  %for mathematics package
\usepackage{mathtools}  %for floor and ceiling
\usepackage{algorithmicx}  %for dynamic algorithm
\usepackage{algorithm}  %algorithm micro
\usepackage{algpseudocode}  %pseudocode commands
\usepackage{wrapfig}  %for wrapping figures around text
\usepackage{multicol}  %for multiple columns floats
\usepackage{enumitem}  %for enumerate numbering
\usepackage{url}   %for writing the url
\usepackage{color}  %for colorred text
\usepackage{tcolorbox}  %for colour box highlighting
\usepackage{listings}  %for code listing

% $$$$$$$$$ new command and theorems self-defined
\newtheorem{theorem}{Theorem}[section]
\newtheorem{corollary}{Corollary}[theorem]
\newtheorem{lemma}[theorem]{Lemma}
\newtheorem*{remark}{Remark}
\theoremstyle{definition}
\newtheorem{definition}{Definition}[section]
\newtheorem{example}{Example}[section]
\newtheorem{notation}{Notation}[section]
\newtheorem{algoalgorithm}{Algorithm}[section]

% $$$$$$$$ set general info $$$$$$$
\title{\textsl{National University of Singapore} \\ \textbf{CS2106 Operating System}\\ Midterm Summary Notes}
\author{\textit{Dong Shaocong} A0148008J}

% $$$$$$$$ package parameter setting $$$$$$$$$

% for title spacing {left}{before}{after}  ----------------------------
\titlespacing\section{0.5pt}{10pt plus 2pt minus 2pt}{2pt plus 2pt minus 1pt}
\titlespacing\subsection{0.5pt}{10pt plus 2pt minus 2pt}{2pt plus 2pt minus 1pt}
\titlespacing\subsubsection{0.5pt}{10pt plus 2pt minus 2pt}{2pt plus 2pt minus 1pt}

% for title font specifications  ------------------------------------------
\titleformat{\section}
  {\normalfont\fontsize{17}{17}\bfseries}
  {\thesection}{1em}{}
  
\titleformat{\subsection}
  {\normalfont\fontsize{15}{15}\bfseries}{\thesection}{1em}{}
  
\titleformat{\subsubsection}
  {\normalfont\fontsize{13}{13}\bfseries}{\thesection}{1em}{}

% declare floor and ceiling functions   ------------------------------------------
\DeclarePairedDelimiter\ceil{\lceil}{\rceil}
\DeclarePairedDelimiter\floor{\lfloor}{\rfloor}

% set the numbering of enumerate to numbers------------------------------------------
\setlist[enumerate]{label*=\arabic*.}
%\setlist{nolistsep}
\newenvironment{myitemize}
{ \begin{itemize}
    \setlength{\itemsep}{5pt}
    \setlength{\parskip}{0pt}
    \setlength{\parsep}{0pt}     }
{ \end{itemize}                  } 
\newenvironment{myenumerate}
{ \begin{enumerate}
    \setlength{\itemsep}{5pt}
    \setlength{\parskip}{0pt}
    \setlength{\parsep}{0pt}     }
{ \end{enumerate}                } 

% $$$$$$$$$ math symbols cheatsheet
% Caligraphic letters: $\mathcal{A}$ 
% Mathbb letters: $\mathbb{A}$
% Mathfrak letters: $\mathfrak{A}$ 
% Math Sans serif letters: $\mathsf{A}$ 
% $$$$$ color text commands ------------
\newcommand{\redtt}[1]{{\color{red}\texttt{#1}}}
\newcommand{\bluett}[1]{{\color{blue}\texttt{#1}}}
\newcommand{\browntt}[1]{{\color{brown}\texttt{#1}}}
\newcommand{\bluebf}[1]{{\color{blue} \huge \textbf{#1}}}
\renewcommand{\emph}[2]{\redtt{#1} \bluebf{#2}}
%-------------------------------------------------------

% $$$$$$$$ start of documents $$$$$$$$$
\begin{document}
\maketitle

\section{Basic Idea}
\begin{definition}{\textbf{Operating System}}
	is a suite (i.e. a collection) of specialised software that:
	\begin{myitemize}
		\item Gives you access to the hardware devices like disk drives, printers, keyboards and monitors.
		\item Controls and allocate system resources like memory and processor time.
		\item Gives you the tools to customise your and tune your system.
	\end{myitemize}
\end{definition}
\begin{example}
	LINUX, OS X (or MAC OS, a variant of UNIX), Windows 8
\end{example}

\begin{tcolorbox}
	\textsf{What are Operating System?} It usually consists of several parts. (\textsf{Onion Model})
	
	\begin{myitemize}
		\item Bootloader – First program run by the system on start-up. Loads remainder of the OS kernel. 
		\begin{myitemize}
			\item On Wintel systems this is found in the Master Boot Record (MBR) on the hard disk.
		\end{myitemize}
		\item Kernel – The part of the OS that runs almost continuously. 
		\item System Programs – Programs provided by the OS to allow:
		\begin{myitemize}
		\item Access to programs.
		\item Configuration of the OS.
		\item System maintenance, etc.
		\end{myitemize}
	\end{myitemize}
		\includegraphics[scale=0.5]{m1/onionModel}
		\centering
\end{tcolorbox}

\begin{tcolorbox}
	\textsf{Abstraction Layer \& Opearating System Structure}
	
	\includegraphics[scale=0.3]{m1/operatingSystemStructure}
		\centering
	\includegraphics[scale=0.4]{m1/abstractionLayerDescription}
		\centering
\end{tcolorbox}

\begin{definition}{\textbf{Boostrapping}}
	\begin{myitemize}
	\item The \textbf{OS is not present in memory} when a system is “cold started”.
	\begin{myitemize}
		\item When a system is first started up, memory is completely empty.
	\end{myitemize}
	\item We start first with a \textbf{bootloader} to get an operating system into memory.
	\begin{myitemize}
	\item Tiny program in the first (few) sector(s) of the hard-disk.
	\item The first sector is generally called the “boot sector” or “master boot record” for this reason.
	\item Job is to load up the main part of the operating system and start it up.
	\end{myitemize}
\end{myitemize}
\end{definition}

\begin{definition}{\textbf{Core}}
	CPU units that can execute processes, because we have much more number of processes than the number of cores, we have to do \textbf{context switching} to share a core very quickly between different processes.
	\begin{myitemize}
		\item Entire sharing must be transparent.
		\item Processes can be suspended and resumed arbitrarily.
	\end{myitemize}
\end{definition}

\begin{definition}{\textbf{Context switching}}
	\begin{myenumerate}
		\item Save the \textsf{context} of the process to be suspended.
		\item Restore the \textsf{context} of the process to be (re)started.
		\item Issues of \textsf{scheduling} to decide which process to run.
	\end{myenumerate}
\end{definition}

\begin{definition}{\textbf{File system}}
	A set of data structures on disk and within the OS kernel memory to organise persistent data.
\end{definition}

\begin{tcolorbox}
\textsf{How OS file system works?}

	\includegraphics[scale=0.3]{m1/fileSystem}
	\centering
\end{tcolorbox}

\begin{tcolorbox}
	\textsf{Hardware Interfaces}
	
	\includegraphics[scale=0.3]{m1/hardwareDevice}
	\centering
	\includegraphics[scale=0.4]{m1/hardwareController}
	\centering
\end{tcolorbox}

\begin{definition}{\textbf{Memory}}
	static/dynamic (\textsf{new, delete, malloc, free}). Memory to store instructions
Memory to store data.
\end{definition}

\begin{tcolorbox}
	\textsf{Memory Management}
	
	\includegraphics[scale=0.3]{m1/memoryManagement}
	\centering
\end{tcolorbox}

\begin{definition}{\textbf{Virtual Memory management}}
\begin{myitemize}
	\item For cost/speed reasons memory is organized in a hierarchy:
	\begin{figure}[h!]
		\includegraphics[scale=0.6]{m1/memoryHirarchy}
		\centering
	\end{figure}
	\item The lowest level is called ''virtual memory'' and is the slowest but cheapest memory.
	\begin{myitemize}
	\item Actually made using hard-disk space!
	\item Allows us to fit much more instructions and data than memory allows!
	\end{myitemize}
\end{myitemize}
\end{definition}

\begin{definition}{\textbf{OS security}}
	\begin{myitemize}
		\item Data (files): Encryption techniques, Access control lists
		\item Resources: Access to the hardware (biometric, passwords, etc), Memory access, File access, etc.
	\end{myitemize}
\end{definition}

\begin{tcolorbox}
	\textsf{Writing an OS (BSD Unix)}
	
	\includegraphics[scale=0.3]{m1/writingOS}
	\centering
\end{tcolorbox}

\begin{definition}{\textbf{Kernel}}
	\begin{myitemize}
		\item \textsf{Monolithic Kernel} (Linux, MS Windows)
		\begin{myitemize}
			\item All major parts of the OS-devices drivers, file systems, IPC, etc, running in ''kernel space'' (an elevated execution mode where certain privileged operations are allowed).
			\item Bits and pieces of the kernel can be loaded and unloaded at runtime (e.g. using ''modprobe'' in Linux)
			\begin{figure}[h!]
				\includegraphics[scale=0.35]{m1/monolithicKernels}
				\centering
			\end{figure}
		\end{myitemize}
		\item \textsf{MicroKernel} (Mac OS)
		\begin{myitemize}
			\item Only the ''main'' part of the kernel is in ''kernel space'' (Contains the important stuff like the scheduler, process management, memory management, etc.)
			\item The other parts of the kernel operate in ''user space'' as system services: The file systems, USB device drivers, Other device drivers.
		\end{myitemize}
	\end{myitemize}
\end{definition}

\begin{tcolorbox}
	\textsf{External View of an OS}
	\begin{myitemize}
		\item The kernel itself is not very useful. (Provides key functionality, but need a way to access all this functionality.)
		\item We need other components:
		\begin{myitemize}
			\item System libraries (e.g. stdio, unistd, etc.)
			\item System services (creat, read, write, ioctl, sbrk, etc.)
			\item OS Configuration (task manager, setup, etc.)
			\item System programs (Xcode, vim, etc.)
			\item Shells (bash, X-Win, Windows GUI, etc.)
			\item Admin tools (User management, disk optimization, etc.)
			\item User applications (Word, Chrome, etc).
		\end{myitemize}
	\end{myitemize}
	
\end{tcolorbox}

\begin{definition}{\textbf{System Calls}}
	calls made to the “Application Program Interface” or API of the OS.
	\begin{myitemize}
		\item UNIX  and similar OS mostly follow the POSIX standard. (Based on C. Programs become more portable.) \textit{POSIX: portable operating system interface for UNIX, minimal set of system calls for application portability between variants of UNIX}.
		\item Windows follows the WinAPI standard. (Windows 7 and earlier provide Win32/Win64, based on C.
Windows 8 provide Win32/Win64 (based on C) and WinRT (based on C++).)
	\end{myitemize}
\end{definition}

\begin{example}{\textbf{User mode + Kernel mode}}
	\begin{myitemize}
		\item Programs (process) run in user mode.
		\item During system calls, running kernel code in kernel mode.
		\item After system call, back to user mode.
	\end{myitemize}
\end{example}

\begin{tcolorbox}
	\textsf{How to switch mode?} Use privilege mode to switching instructions:
	\begin{myitemize}
		\item syscall instruction
		\item software interrupt - instruction which raises specific interrupt from software.
	\end{myitemize}
\end{tcolorbox}

\begin{example}{\textbf{LINUX system call}}
	\begin{myitemize}
		\item User mode: (outside kernel)
		\begin{myitemize}
			\item C function wrapper (eg. \textbf{getpid()}) for every system call in C library.
			\item assembler code to setup the system call no, arguments
			\item trap to kernel	
		\end{myitemize}
		\item Kernel mode: (inside kernel)
		\begin{myitemize}
			\item dispatch to correct routine
			\item check arguments for errors (eg. invalid argument, invalid address, security violation)
			\item do requested service
			\item return from kernel trap to user mode
		\end{myitemize}
		\item User mode: (Outside kernel)
		\begin{myitemize}
			\item returns to C wrapper - check for error return values
		\end{myitemize}
	\end{myitemize}
\end{example}



\section{Process Management}


























% $$$$$$$$$$$$$$$$$$$$$$$$$$$$$$$$$$ %

\newpage
\begin{myitemize}
	\item 123
\end{myitemize}
{\color{red} red } \textsf{sftext} \texttt{tttext}
\begin{tcolorbox} 
Penghao is the best \LaTeX \, writer.
\end{tcolorbox}
\begin{tcolorbox}
	\begin{lstlisting}
		def func():
			print("Penghao is cool!")
	\end{lstlisting}
\end{tcolorbox}
\noindent{}\\\noindent$\wp\wp\wp\wp\wp\wp\wp\wp\wp\wp\wp\wp\wp\wp\wp\wp\wp\wp\wp\wp\wp\wp\wp\wp\wp\wp\wp\wp\wp\wp\wp\wp\wp\wp\wp$\\
{}\\
\lipsum[1]

% ---- important examples ----%
\vspace{-\topsep}
\begin{enumerate}
	\item 123
	\item 123
	\item 123
\end{enumerate}
\vspace{-\topsep}

\begin{center}
 \begin{tabular}{|| c || c | c | c |c |c ||} 
 \hline
  & M1 & M2 & M3 & M4 & M5 \\ [0.5ex] 
 \hline\hline
 M1 & 0 & 108 & 180 & 228 & 396 \\ 
 \hline
 M2 &  & 0 & 72 & 168 & 288 \\
 \hline
 M3 &  &  & 0 & 48 & 144 \\
 \hline
 M4 &  &  &  & 0 & 128 \\
 \hline
 M5 &  &  & &  & 0   \\ [0.5ex] 
 \hline
\end{tabular}
\end{center}

\begin{algorithm}
\caption{title}
%\begin{multicols}{2}
\begin{algorithmic}[1]
\State 123
\end{algorithmic}
%\end{multicols}
\end{algorithm}

\begin{thebibliography}{9}
 
\bibitem{einstein} 
Albert Einstein. 
\textit{Zur Elektrodynamik bewegter K{\"o}rper}. (German) 
[\textit{On the electrodynamics of moving bodies} \url{www.google.com.sg}]. 
Annalen der Physik, 322(10):891–921, 1905.
\end{thebibliography}

\end{document}